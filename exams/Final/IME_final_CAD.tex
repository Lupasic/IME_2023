\documentclass[12pt]{article}

\usepackage{geometry}
 \geometry{
 a4paper,
 total={170mm,257mm},
 left=20mm,
 right=10mm,
 top=45mm,
 bottom=20mm,
 headheight=75pt
 }
% \usepackage[utf8x]{inputenc}
\usepackage{fontspec}
\setmainfont{Open Sans}
\setsansfont{Noto Sans}
\usepackage{graphicx}
\usepackage{forloop}
\usepackage{subcaption}
\usepackage{url}       % `\url`s
\usepackage{floatrow}
\usepackage{hyperref}
\usepackage{cleveref}
\graphicspath{{resources_CAD/}}

\usepackage{fancyhdr}
\pagestyle{fancy}


\renewcommand{\headrulewidth}{0pt}
\fancyhead[C]{\LARGE <<Introduction to Mechanical Engineering>> \\ \textbf{Final Exam} \\ \textit{CAD part} \\ Variant: \thepage}
\fancyfoot[]{}


\newcommand\pic[1]{(\cref{#1})} %Где нужно сослаться на рисунок

\hypersetup{
    colorlinks=true,
    linkcolor=blue,
    urlcolor=cyan,
    }

% The preamble ends with the command \begin{document}
\newcounter{themenumber}
\begin{document}
\forloop{themenumber}{1}{\value{themenumber} < 11}{
\begin{enumerate}
    \item  Make a CAD model of the blueprint, which provided below. (15 score)
    \item  Make an assembly 
    \item (2 extra score) Make the same blueprints (without dimensions), based on your CAD model.
    \item (3 extra score) Perform the stress analysis of the detail. All forces and fix supports are on the picture. Material --- Steel. You have to show the stress and strain diagrams and explain what happens to the parts after such a load. 
\end{enumerate}

\begin{figure}[H]
    \centering\includegraphics[height=18cm,width=1\textwidth,keepaspectratio]{var\arabic{themenumber}.png}
\end{figure}

\newpage

}
\end{document}