% \documentclass[aspectratio=169,notes]{beamer}
\documentclass[aspectratio=169]{beamer}
\usetheme[faculty=phil]{fibeamer}
\usepackage{polyglossia}
\setmainlanguage{english} %% main locale instead of `english`, you
%% can typeset the presentation in either Czech or Slovak,
%% respectively.
\setotherlanguages{russian} %% The additional keys allow
%%
%%   \begin{otherlanguage}{czech}   ... \end{otherlanguage}
%%   \begin{otherlanguage}{slovak}  ... \end{otherlanguage}
%%
%% These macros specify information about the presentation
\title[IME]{Introduction to Mechanical Engineering, CAE DYN 2} %% that will be typeset on the
\subtitle{Motion Analysis, Part 2
\\ \  \\ \ 
    } %% title page.
\author{Oleg Bulichev}
%% These additional packages are used within the document:
\usepackage{ragged2e}  % `\justifying` text
\usepackage{booktabs}  % Tables
\usepackage{tabularx}
\usepackage{tikz}      % Diagrams
\usetikzlibrary{calc, shapes, backgrounds}
\usepackage{amsmath, amssymb}
\usepackage{url}       % `\url`s
\usepackage{listings}  % Code listings
% \usepackage{subfigure}
\usepackage{floatrow}
\usepackage{subcaption}
\usepackage{mathtools}
\usepackage{todonotes}
\usepackage{fontspec}
\usepackage{multicol}
\usepackage{pdfpages}
\usepackage{wrapfig}
\usepackage{animate}
\usepackage{booktabs}
\usepackage{multirow}
% \usepackage{graphicx}
\usepackage{colortbl}

\graphicspath{{resources/}}
\frenchspacing

\setbeamertemplate{caption}[numbered]
\usetikzlibrary{graphs}

% \usepackage[backend=biber,style=ieee,autocite=footnote]{biblatex}
% \addbibresource{biblio.bib}
% \DefineBibliographyStrings{english}{%
%   bibliography = {References},}

\newcommand{\oleg}[2][] {\todo[color=red, #1] {OLEG:\\ #2}}
\newcommand{\fbckg}[1]{\usebackgroundtemplate{\includegraphics[width=\paperwidth]{#1}}}%frame background

\usepackage[framemethod=TikZ]{mdframed}
\newcommand{\dbox}[1]{
\begin{mdframed}[roundcorner=3pt, backgroundcolor=yellow, linewidth=0]
\vspace{1mm}
{#1}
\vspace{1mm}
\end{mdframed}
}

\begin{document}
\setlength{\abovedisplayskip}{0pt}
\setlength{\belowdisplayskip}{0pt}
\setlength{\abovedisplayshortskip}{0pt}
\setlength{\belowdisplayshortskip}{0pt}

\fbckg{fibeamer/figs/title_page.png}
\frame[c]{\setcounter{framenumber}{0}
    \usebeamerfont{title}%
    \usebeamercolor[fg]{title}%
    \begin{minipage}[b][6.5\baselineskip][b]{\textwidth}%
        \textcolor{black}{\raggedright\inserttitle}
    \end{minipage}
    % \vskip-1.5\baselineskip

    \usebeamerfont{subtitle}%
    \usebeamercolor[fg]{framesubtitle}%
    \begin{minipage}[b][3\baselineskip][b]{\textwidth}
        \raggedright%
        \insertsubtitle%
    \end{minipage}
    \vskip.25\baselineskip
}
%   \frame[c]{\maketitle}

\fbckg{fibeamer/figs/common.png}

\note{\scriptsize
\ 
}

\begin{frame}[c]{Guideline}
\framesubtitle{}
    \centering \LARGE You have 2 weeks to solve \textit{HW\_CAE\_DYN2},
    
    using reference material and lecturer supervising.
\end{frame}

\begin{frame}[t]{Reference Material}
\framesubtitle{}
\begin{enumerate}
    \item \textit{labs/CAE\_DYN2/CAE\_DYN2\_explanatory\_material.pdf} --- main material. It was made by Oleg's BMSTU supervisor. I adapted it for our purpose (erase unnecessary material).
    \item \href{https://docs.sw.siemens.com/en-US/doc/289054037/PL20201105153211099.motion/id563946}{Simcenter 3D Motion Documentation (official)}
    \item \href{https://docs.sw.siemens.com/en-US/doc/289054037/PL20201105153211099.xid1391716/xid1964458}{Motion Analysis Tutorials by Siemens (official)}
    \item \href{https://www.youtube.com/playlist?list=PLY8N5WFx1MGAsxH7G49ey37QC\_nFtQ72E}{Motion Analysis video tutorials 1}
    \item \href{https://www.youtube.com/playlist?list=PL0B-6-amsUZCDepzFMcGETf9pb9CSiqTL}{Motion Analysis video tutorials 2}
\end{enumerate}
\end{frame}

\fbckg{fibeamer/figs/last_page.png}
\frame[plain]{}

\end{document}