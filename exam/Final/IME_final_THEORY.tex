\documentclass[12pt]{article}

\usepackage{geometry}
 \geometry{
 a4paper,
 total={170mm,257mm},
 left=20mm,
 right=10mm,
 top=10mm,
 bottom=20mm,
 }
% \usepackage[utf8x]{inputenc}
\usepackage{fontspec}
\setmainfont{Open Sans}
\setsansfont{Noto Sans}
\usepackage{graphicx}
\usepackage{forloop}
\usepackage{subcaption}
\usepackage{url}       % `\url`s
\usepackage{floatrow}
\usepackage{hyperref}
\usepackage{cleveref}
\graphicspath{{resources_CAD/}}

\usepackage{fancyhdr}
\pagestyle{fancy}


\renewcommand{\headrulewidth}{0pt}
\fancyhead[C]{}
\fancyfoot[]{}


\newcommand\pic[1]{(\cref{#1})} %Где нужно сослаться на рисунок

\hypersetup{
    colorlinks=true,
    linkcolor=blue,
    urlcolor=cyan,
    }

\newcommand\ttask[2]{
	\begin{center}
		\LARGE <<Introduction to Mechanical Engineering>> \\ \textbf{Final Exam} \\ \textit{Theory part} \\ Variant: #1
	\end{center}
#2
}

\def\introTM{Lower and higher kinematic pairs. Examples. Form and forced closed joints. The idea besides calculation mechanism DoF.}
\def\typeD{Common types of drives (at least 5) Examples. Prof and cons.}
\def\SPM{Types of synthesis of a mechanism. 3 general ways of solving methods.}
\def\MS{What the key aspects should we consider during the motor chosing? The general guideline of the motor selection.}
\def\LJC{What types of detachable connections do you know (at least 4)? Examples. Prof and cons.}
\def\LJCC{What types of permanent connections do you know (at least 4)? Examples. Prof and cons.}
\def\LJCCC{Bearings. Types. Prof and cons. How to mount and dismount them. The idea besides locating and floating bearings.}
\def\CD{Screw types. Multisided screws, prof and cons. Type of drills. Type of holes. How to distinguish them on a blueprints?}
\def\EM{Stress-strain curve. What the idea besides it? How can we modify it for a material?}
\def\EMM{Iron - Carbon plot. Why do we need this plot? Aluminum and titanium. Prof and cons.}
\def\DTM{Types of manufacturing. At least 1 example for each type. Prof and cons.}
\def\BP{3D printer. How does the printer print (FDM)? The guideline of printing process. The best practices for orienting part on a printer table.}
\def\SoM{Main difference between Theoretical Mechanics and Strength of Material courses in terms of concept. Common types of loads.}
\def\FDM{ODE and PDE, difference. Boundary and Initial value problems. The main idea besides Finite Difference and Finite Element Methods.}

\begin{document}
\ttask{1}{\begin{enumerate}
	\item \introTM
	\item \BP
\end{enumerate}}

\ttask{2}{\begin{enumerate}
	\item \typeD
	\item \SoM
\end{enumerate}}

\ttask{3}{\begin{enumerate}
	\item \SPM
	\item \EM
\end{enumerate}}

\ttask{4}{\begin{enumerate}
	\item \MS
	\item \CD
\end{enumerate}}

\ttask{5}{\begin{enumerate}
	\item \LJC
	\item \EM
\end{enumerate}}

\ttask{6}{\begin{enumerate}
	\item \LJCC
	\item \FDM
\end{enumerate}}

\ttask{7}{\begin{enumerate}
	\item \LJCCC
	\item \FDM
\end{enumerate}}

\ttask{8}{\begin{enumerate}
	\item \CD
	\item \typeD
\end{enumerate}}

\ttask{9}{\begin{enumerate}
	\item \EM
	\item \MS
\end{enumerate}}

\ttask{10}{\begin{enumerate}
	\item \EMM
	\item \LJCC
\end{enumerate}}

\ttask{11}{\begin{enumerate}
	\item \DTM
	\item \introTM
\end{enumerate}}

\ttask{12}{\begin{enumerate}
	\item \BP
	\item \CD
\end{enumerate}}

\ttask{13}{\begin{enumerate}
	\item \SoM
	\item \LJCCC
\end{enumerate}}

\ttask{14}{\begin{enumerate}
	\item \FDM
	\item \LJCCC
\end{enumerate}}

\ttask{15}{\begin{enumerate}
	\item \introTM
	\item \SPM
\end{enumerate}}

\ttask{16}{\begin{enumerate}
	\item \typeD
	\item \LJC
\end{enumerate}}

\ttask{17}{\begin{enumerate}
	\item \SPM
	\item \DTM
\end{enumerate}}

\ttask{18}{\begin{enumerate}
	\item \MS
	\item \FDM
\end{enumerate}}


\end{document}