\documentclass{article}

\usepackage{geometry}
 \geometry{
 a4paper,
 total={170mm,257mm},
 left=20mm,
 right=10mm,
 top=10mm,
 bottom=20mm,
 }
% \usepackage[utf8x]{inputenc}
\usepackage{fontspec}
\setmainfont{Open Sans}
\setsansfont{Noto Sans}
\usepackage{graphicx}
\usepackage{url}       % `\url`s
\usepackage{floatrow}
\usepackage{hyperref}
\graphicspath{{resources/}}

\title{IME HW CAD ASM 1} % Sets article title
\author{} % Sets authors name
\date{}

\hypersetup{
    colorlinks=true,
    linkcolor=blue,
    urlcolor=cyan,
    }

\newcommand\ttask[3] 
 {
	\section*{Variant #1}
	\textbf{Assembly description}: \\
	\underline{ENG}: #2 \smallskip \\ 
	\underline{RUS}: #3
	\begin{figure}[H]
		\centering\includegraphics[height=23cm,width=1\textwidth,keepaspectratio]{#1.png}
		\label{fig:#1.png}
	\end{figure}
	\newpage
 }

% The preamble ends with the command \begin{document}
\begin{document} % All begin commands must be paired with an end command somewhere
	\maketitle % creates title using information in preamble (title, author, date)
    
	%New section is created
	\section{Task description}
	\begin{enumerate}
		\item Create all 3D CAD models, based on provided blueprints.
		\item If it is needed (it's written in task personal description), download and import a step file of needed screw. It can be downloaded from \href{https://www.mcmaster.com/}{this site}. \textbf{You should find a GOST alternative in the internet.} 
		\item Make an assembly according to the scheme provided.
	\end{enumerate}
	\textbf{Artifacts}: .zip file, contains all details in .prt and assembly file <<\url{IME_HW_CAD_ASM1_assembly.prt}>>

	\newpage
	\ttask{1}{Plate \textbf{2} is attached to the clamp \textbf{1} screw \textbf{5} (M12x30 GOST 17473-80) and a nut 6 (M12 GOST 5915-70). The base \textbf{3} is connected to the clamp with two screws \textbf{4} (M8x16 GOST 1491-80).}{Тарелка \textbf{2} прикреплена к фиксатору \textbf{1} винтом \textbf{5} (М12х30 ГОСТ 17473-80) и гайкой \textbf{6} (М12 ГОСТ 5915-70). Основание \textbf{3} соединено с фиксатором двумя винтами \textbf{4} (M8x16 ГОСТ 1491-80).}
	
	\ttask{2}{Plate \textbf{2} and three plates \textbf{3} are connected to the base \textbf{1} with two screws \textbf{4} (M8x45 GOST 1491-80).}{Пластина \textbf{2} и три пластины \textbf{3} соединяются с основанием \textbf{1} двумя винтами \textbf{4} (М8х45 ГОСТ 1491-80).}

	\ttask{3}{Plate \textbf{1} is attached to the flanges \textbf{2} and \textbf{4} with four screws \textbf{5} (M6x12 GOST 17473-80). Then these three parts in the cathedral are inserted into the body \textbf{3} and connected with screw \textbf{6} (M8x10 GOST 17475-80).}{Пластина \textbf{1} крепится к фланцам \textbf{2} и \textbf{4} четырьмя винтами \textbf{5} (М6х12 ГОСТ 17473-80). Затем эти три детали в соборе вкладываются в корпус \textbf{3} и соединяются винтом \textbf{6} (М8х10 ГОСТ 17475-80).}

	\ttask{4}{Body \textbf{2} is connected to the base \textbf{1} with two screws \textbf{4} (M8x35 GOST 1491-80). The cover \textbf{3} is attached to the housing with two screws \textbf{5} (M8x25 GOST 17473-80).}{Корпус \textbf{2} соединяется с основанием \textbf{1} двумя винтами \textbf{4} (М8х35 ГОСТ 1491-80). Крышка \textbf{3} крепится к корпусу двумя винтами \textbf{5} (М8х25 ГОСТ 17473-80).}

	\ttask{5}{Body \textbf{1} is connected to the ring \textbf{2} with three screws \textbf{4} (M10x25 GOST 17473-80). The cover \textbf{3} is mounted on the housing, closing the hole in the housing.}{Корпус \textbf{1} соединяется с кольцом \textbf{2} тремя винтами \textbf{4} (М10х25 ГОСТ 17473-80). Крышка \textbf{3} устанавливается на корпус, закрывая отверстие в корпусе.}

	\ttask{6}{Two brackets \textbf{3} are installed in the housing slot \textbf{1} and fastened with two screws \textbf{4} (M8x20 GOST 17473-80). The cover \textbf{2} is connected to the housing with four screws \textbf{5} (M8x20 GOST 17475-80).}{В пазу корпуса \textbf{1} устанавливается две скобы \textbf{3} и крепятся двумя винтами \textbf{4} (M8x20 ГОСТ 17473-80). Крышка \textbf{2} соединяется с корпусом четырьмя винтами \textbf{5} (M8x20 ГОСТ 17475-80).}

\end{document} % This is the end of the document